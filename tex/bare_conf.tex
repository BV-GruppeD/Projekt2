% !TeX spellcheck = de_DE

%% vorl_bv
%% modified by Jan Leif Hoffmann, 2017-11-20
%% further comments reflect the original bare_conf.tex file contents
%% and are partly not applicable to the purpose at hand.
%% based on
%% bare_conf.tex
%% V1.4
%% 2012/12/27
%% by Michael Shell
%% See:
%% http://www.michaelshell.org/
%% for current contact information.
%%
%% This is a skeleton file demonstrating the use of IEEEtran.cls
%% (requires IEEEtran.cls version 1.8 or later) with an IEEE conference paper.
%%
%% Support sites:
%% http://www.michaelshell.org/tex/ieeetran/
%% http://www.ctan.org/tex-archive/macros/latex/contrib/IEEEtran/
%% and
%% http://www.ieee.org/

%%*************************************************************************
%% Legal Notice:
%% This code is offered as-is without any warranty either expressed or
%% implied; without even the implied warranty of MERCHANTABILITY or
%% FITNESS FOR A PARTICULAR PURPOSE!
%% User assumes all risk.
%% In no event shall IEEE or any contributor to this code be liable for
%% any damages or losses, including, but not limited to, incidental,
%% consequential, or any other damages, resulting from the use or misuse
%% of any information contained here.
%%
%% All comments are the opinions of their respective authors and are not
%% necessarily endorsed by the IEEE.
%%
%% This work is distributed under the LaTeX Project Public License (LPPL)
%% ( http://www.latex-project.org/ ) version 1.3, and may be freely used,
%% distributed and modified. A copy of the LPPL, version 1.3, is included
%% in the base LaTeX documentation of all distributions of LaTeX released
%% 2003/12/01 or later.
%% Retain all contribution notices and credits.
%% ** Modified files should be clearly indicated as such, including  **
%% ** renaming them and changing author support contact information. **
%%
%% File list of work: IEEEtran.cls, IEEEtran_HOWTO.pdf, bare_adv.tex,
%%                    bare_conf.tex, bare_jrnl.tex, bare_jrnl_compsoc.tex,
%%                    bare_jrnl_transmag.tex
%%*************************************************************************

% *** Authors should verify (and, if needed, correct) their LaTeX system  ***
% *** with the testflow diagnostic prior to trusting their LaTeX platform ***
% *** with production work. IEEE's font choices can trigger bugs that do  ***
% *** not appear when using other class files.                            ***
% The testflow support page is at:
% http://www.michaelshell.org/tex/testflow/



% Note that the a4paper option is mainly intended so that authors in
% countries using A4 can easily print to A4 and see how their papers will
% look in print - the typesetting of the document will not typically be
% affected with changes in paper size (but the bottom and side margins will).
% Use the testflow package mentioned above to verify correct handling of
% both paper sizes by the user's LaTeX system.
%
% Also note that the "draftcls" or "draftclsnofoot", not "draft", option
% should be used if it is desired that the figures are to be displayed in
% draft mode.
%
\documentclass[conference,a4paper, 11pt]{IEEEtran}
% Add the compsoc option for Computer Society conferences.
%
% If IEEEtran.cls has not been installed into the LaTeX system files,
% manually specify the path to it like:
% \documentclass[conference]{../sty/IEEEtran}





% Some very useful LaTeX packages include:
% (uncomment the ones you want to load)


% *** MISC UTILITY PACKAGES ***
%
%\usepackage{ifpdf}
% Heiko Oberdiek's ifpdf.sty is very useful if you need conditional
% compilation based on whether the output is pdf or dvi.
% usage:
% \ifpdf
%   % pdf code
% \else
%   % dvi code
% \fi
% The latest version of ifpdf.sty can be obtained from:
% http://www.ctan.org/tex-archive/macros/latex/contrib/oberdiek/
% Also, note that IEEEtran.cls V1.7 and later provides a builtin
% \ifCLASSINFOpdf conditional that works the same way.
% When switching from latex to pdflatex and vice-versa, the compiler may
% have to be run twice to clear warning/error messages.






% *** CITATION PACKAGES ***
%
\usepackage{cite}
% cite.sty was written by Donald Arseneau
% V1.6 and later of IEEEtran pre-defines the format of the cite.sty package
% \cite{} output to follow that of IEEE. Loading the cite package will
% result in citation numbers being automatically sorted and properly
% "compressed/ranged". e.g., [1], [9], [2], [7], [5], [6] without using
% cite.sty will become [1], [2], [5]--[7], [9] using cite.sty. cite.sty's
% \cite will automatically add leading space, if needed. Use cite.sty's
% noadjust option (cite.sty V3.8 and later) if you want to turn this off
% such as if a citation ever needs to be enclosed in parenthesis.
% cite.sty is already installed on most LaTeX systems. Be sure and use
% version 4.0 (2003-05-27) and later if using hyperref.sty. cite.sty does
% not currently provide for hyperlinked citations.
% The latest version can be obtained at:
% http://www.ctan.org/tex-archive/macros/latex/contrib/cite/
% The documentation is contained in the cite.sty file itself.






% *** GRAPHICS RELATED PACKAGES ***
%
\ifCLASSINFOpdf
  \usepackage[pdftex]{graphicx}
  % declare the path(s) where your graphic files are
  % \graphicspath{{../pdf/}{../jpeg/}}
  % and their extensions so you won't have to specify these with
  % every instance of \includegraphics
  % \DeclareGraphicsExtensions{.pdf,.jpeg,.png}
\else
  % or other class option (dvipsone, dvipdf, if not using dvips). graphicx
  % will default to the driver specified in the system graphics.cfg if no
  % driver is specified.
  % \usepackage[dvips]{graphicx}
  % declare the path(s) where your graphic files are
  % \graphicspath{{../eps/}}
  % and their extensions so you won't have to specify these with
  % every instance of \includegraphics
  % \DeclareGraphicsExtensions{.eps}
\fi
% graphicx was written by David Carlisle and Sebastian Rahtz. It is
% required if you want graphics, photos, etc. graphicx.sty is already
% installed on most LaTeX systems. The latest version and documentation
% can be obtained at:
% http://www.ctan.org/tex-archive/macros/latex/required/graphics/
% Another good source of documentation is "Using Imported Graphics in
% LaTeX2e" by Keith Reckdahl which can be found at:
% http://www.ctan.org/tex-archive/info/epslatex/
%
% latex, and pdflatex in dvi mode, support graphics in encapsulated
% postscript (.eps) format. pdflatex in pdf mode supports graphics
% in .pdf, .jpeg, .png and .mps (metapost) formats. Users should ensure
% that all non-photo figures use a vector format (.eps, .pdf, .mps) and
% not a bitmapped formats (.jpeg, .png). IEEE frowns on bitmapped formats
% which can result in "jaggedy"/blurry rendering of lines and letters as
% well as large increases in file sizes.
%
% You can find documentation about the pdfTeX application at:
% http://www.tug.org/applications/pdftex





% *** MATH PACKAGES ***
%
\usepackage[cmex10]{amsmath}
% A popular package from the American Mathematical Society that provides
% many useful and powerful commands for dealing with mathematics. If using
% it, be sure to load this package with the cmex10 option to ensure that
% only type 1 fonts will utilized at all point sizes. Without this option,
% it is possible that some math symbols, particularly those within
% footnotes, will be rendered in bitmap form which will result in a
% document that can not be IEEE Xplore compliant!
%
% Also, note that the amsmath package sets \interdisplaylinepenalty to 10000
% thus preventing page breaks from occurring within multiline equations. Use:
%\interdisplaylinepenalty=2500
% after loading amsmath to restore such page breaks as IEEEtran.cls normally
% does. amsmath.sty is already installed on most LaTeX systems. The latest
% version and documentation can be obtained at:
% http://www.ctan.org/tex-archive/macros/latex/required/amslatex/math/





% *** SPECIALIZED LIST PACKAGES ***
%
%\usepackage{algorithmic}
% algorithmic.sty was written by Peter Williams and Rogerio Brito.
% This package provides an algorithmic environment fo describing algorithms.
% You can use the algorithmic environment in-text or within a figure
% environment to provide for a floating algorithm. Do NOT use the algorithm
% floating environment provided by algorithm.sty (by the same authors) or
% algorithm2e.sty (by Christophe Fiorio) as IEEE does not use dedicated
% algorithm float types and packages that provide these will not provide
% correct IEEE style captions. The latest version and documentation of
% algorithmic.sty can be obtained at:
% http://www.ctan.org/tex-archive/macros/latex/contrib/algorithms/
% There is also a support site at:
% http://algorithms.berlios.de/index.html
% Also of interest may be the (relatively newer and more customizable)
% algorithmicx.sty package by Szasz Janos:
% http://www.ctan.org/tex-archive/macros/latex/contrib/algorithmicx/




% *** ALIGNMENT PACKAGES ***
%
%\usepackage{array}
% Frank Mittelbach's and David Carlisle's array.sty patches and improves
% the standard LaTeX2e array and tabular environments to provide better
% appearance and additional user controls. As the default LaTeX2e table
% generation code is lacking to the point of almost being broken with
% respect to the quality of the end results, all users are strongly
% advised to use an enhanced (at the very least that provided by array.sty)
% set of table tools. array.sty is already installed on most systems. The
% latest version and documentation can be obtained at:
% http://www.ctan.org/tex-archive/macros/latex/required/tools/


% IEEEtran contains the IEEEeqnarray family of commands that can be used to
% generate multiline equations as well as matrices, tables, etc., of high
% quality.




% *** SUBFIGURE PACKAGES ***
%\ifCLASSOPTIONcompsoc
%  \usepackage[caption=false,font=normalsize,labelfont=sf,textfont=sf]{subfig}
%\else
%  \usepackage[caption=false,font=footnotesize]{subfig}
%\fi
% subfig.sty, written by Steven Douglas Cochran, is the modern replacement
% for subfigure.sty, the latter of which is no longer maintained and is
% incompatible with some LaTeX packages including fixltx2e. However,
% subfig.sty requires and automatically loads Axel Sommerfeldt's caption.sty
% which will override IEEEtran.cls' handling of captions and this will result
% in non-IEEE style figure/table captions. To prevent this problem, be sure
% and invoke subfig.sty's "caption=false" package option (available since
% subfig.sty version 1.3, 2005/06/28) as this is will preserve IEEEtran.cls
% handling of captions.
% Note that the Computer Society format requires a larger sans serif font
% than the serif footnote size font used in traditional IEEE formatting
% and thus the need to invoke different subfig.sty package options depending
% on whether compsoc mode has been enabled.
%
% The latest version and documentation of subfig.sty can be obtained at:
% http://www.ctan.org/tex-archive/macros/latex/contrib/subfig/




% *** FLOAT PACKAGES ***
%
%\usepackage{fixltx2e}
% fixltx2e, the successor to the earlier fix2col.sty, was written by
% Frank Mittelbach and David Carlisle. This package corrects a few problems
% in the LaTeX2e kernel, the most notable of which is that in current
% LaTeX2e releases, the ordering of single and double column floats is not
% guaranteed to be preserved. Thus, an unpatched LaTeX2e can allow a
% single column figure to be placed prior to an earlier double column
% figure. The latest version and documentation can be found at:
% http://www.ctan.org/tex-archive/macros/latex/base/


%\usepackage{stfloats}
% stfloats.sty was written by Sigitas Tolusis. This package gives LaTeX2e
% the ability to do double column floats at the bottom of the page as well
% as the top. (e.g., "\begin{figure*}[!b]" is not normally possible in
% LaTeX2e). It also provides a command:
%\fnbelowfloat
% to enable the placement of footnotes below bottom floats (the standard
% LaTeX2e kernel puts them above bottom floats). This is an invasive package
% which rewrites many portions of the LaTeX2e float routines. It may not work
% with other packages that modify the LaTeX2e float routines. The latest
% version and documentation can be obtained at:
% http://www.ctan.org/tex-archive/macros/latex/contrib/sttools/
% Do not use the stfloats baselinefloat ability as IEEE does not allow
% \baselineskip to stretch. Authors submitting work to the IEEE should note
% that IEEE rarely uses double column equations and that authors should try
% to avoid such use. Do not be tempted to use the cuted.sty or midfloat.sty
% packages (also by Sigitas Tolusis) as IEEE does not format its papers in
% such ways.
% Do not attempt to use stfloats with fixltx2e as they are incompatible.
% Instead, use Morten Hogholm'a dblfloatfix which combines the features
% of both fixltx2e and stfloats:
%
% \usepackage{dblfloatfix}
% The latest version can be found at:
% http://www.ctan.org/tex-archive/macros/latex/contrib/dblfloatfix/




% *** PDF, URL AND HYPERLINK PACKAGES ***
%
%\usepackage{url}
% url.sty was written by Donald Arseneau. It provides better support for
% handling and breaking URLs. url.sty is already installed on most LaTeX
% systems. The latest version and documentation can be obtained at:
% http://www.ctan.org/tex-archive/macros/latex/contrib/url/
% Basically, \url{my_url_here}.




% *** Do not adjust lengths that control margins, column widths, etc. ***
% *** Do not use packages that alter fonts (such as pslatex).         ***
% There should be no need to do such things with IEEEtran.cls V1.6 and later.
% (Unless specifically asked to do so by the journal or conference you plan
% to submit to, of course. )


% correct bad hyphenation here
\hyphenation{Glei-chun-gen}

\newcommand{\imageWithHgram}[3] {
  \begin{figure}[h]
  	\centering
  	\subcaptionbox{Bild}{\includegraphics[width=0.40\columnwidth]{example_pictures/#1.png}}%
  	\hfill % <-- Seperation
  	\subcaptionbox{Histogramm}{\includegraphics[width=0.45\columnwidth]{example_pictures/#1_hgram.png}}%
  	\caption{#2}
  	\label{#3}
  \end{figure}
}

% German umlauts etc.
% if not working, try
% \usepackage[ansinew]{inputenc}
\usepackage[utf8]{inputenc}
\usepackage[T1]{fontenc}
\usepackage[ngerman]{babel}
\usepackage{subcaption}
\captionsetup{compatibility=false}

\begin{document}
%
% paper title
% can use linebreaks \\ within to get better formatting as desired
% Do not put math or special symbols in the title.
\title{Projekt 2: Kontrastanpassung}


% author names and affiliations
% use a multiple column layout for up to three different
% affiliations
\author{\IEEEauthorblockN{Patrick Schlüter}
\IEEEauthorblockA{
	Hauptstraße 1,
	 32657 Lemgo\\
E-Mail: patrick.schlueter@stud.hs-owl.de
% ==================== Disclamer ====================\\
 % Habe noch keine Literatur. \\
 % Wollte sehen, welche ihr nutzt und dann noch reinbauen
}
}

% make the title area
\maketitle

% As a general rule, do not put math, special symbols or citations
% in the abstract
\begin{abstract}
	Bilder mit sehr geringem Kontrast sind für Menschen nur schwer zu erkennen. Wir haben einen Algorithmus zur automatischen Kontrastanpassung implementiert und auf drei Testbilder angewendet. Dann haben wir die Erkennbarkeit der angepassten Bildern mit den Originalen verglichen.
  Unser Ergebnis ist, dass die Kontrastanpassung tatsächlich die Erkennbarkeit von Bildern verbessert, allerdingss nur für Bilder welche einen sehr geringen Kontrast hatten.
\end{abstract}

% no keywords




% For peer review papers, you can put extra information on the cover
% page as needed:
% \ifCLASSOPTIONpeerreview
% \begin{center} \bfseries EDICS Category: 3-BBND \end{center}
% \fi
%
% For peerreview papers, this IEEEtran command inserts a page break and
% creates the second title. It will be ignored for other modes.
\IEEEpeerreviewmaketitle



\section{Einleitung}
% no \IEEEPARstart
Kontrastanpassung ist ein grundlegender Schritt der Bildvorverarbeitung [?citation needed?].
Viele Algorithmen der Bildverarbeitung funktionieren nicht auf Bildern welche einen zu geringen Kontrast haben. Aber auch Menschen können Bilder mit zu geringem Kontrast nur schlecht erkennen. Das Ziel dieser Arbeit ist es mittels einer Kontrastanpassung drei uns gegebene Testbilder zu bearbeiten und zu überprüfen ob die Anpassung die Bilder verbessert hat.
TODO: vorausschauenden Gliederung?

\section{Ansatz/Verfahren}
Wir haben lineare Kontrstanpassung gew\"ahlt. Dazu haben wir die Gl.~\ref{eq:Kontrast} auf jeden Pixel einzeln angewendet. Hierbei sind $q_{min}$ und $q_{max}$ die Minimal- bzw. Maximalwerte der Pixel im Originalbild, $p_{min}$ und $p_{max}$ sind die gewünschten Minimal- bzw. Maximalwerte der Pixel im Ausgangsbild.
\begin{equation}
	\label{eq:Kontrast}
newPixel=(oldPixel - q_{min}) \cdot \frac{p_{max} - p_{min}}{q_{max} - q_{min}} + p_{min}
\end{equation}

Um unseren Ansatz zu testen haben wir einige Testbilder bekommen, bei welchen wir den Kontrast maximieren sollten.
Daher haben wird die Formel auf das Bild mit den Parametern $new_{min}=0$ und $new_{max}=255$ angewendet.


\section{Implementierung}
Der Algorithmus wurde als Plugin für ImageJ implementiert.
Die Hauptklasse unseres Codes ist \textit{Kontrastanpassung\_PlugIn}. Diese Klasse implementiert die \textit{PlugInFilter} Schnittstelle von ImageJ um als Plugin genutzt werden zu können. Wenn das Plugin auf ein Bild angewand wird, so wird die \textit{run} Methode mit Informationen über das Bild aufgerufen. Diese erstellt zuerst einen \textit{ContrastAdaptionUserDialog}, welcher ein Dialogfenster mit Schiebereglern für den minimalen Pixelwert, den maximalen Pixelwert und die Sättigung des Ausgangsbildes enthält. Dieser Dialog wird angezeigt bis der Nutzer seine Werte eingegeben und bestätigt hat. TODO weniger detail, fertig stellen

\section{Ergebnisse}
\subsection{Testbild: GruppeDBild}
Das erste Testbild war Abb.~\ref{fig:gruppe_d_orig}, auf welchem nur eine schwarze Fläche zu sehen ist. Anhand des Histogramms kann man sehen, dass der Kontrast sehr gering ist. Daher eignet es sich theoretisch gut für eine Kontrastanpassung.
\imageWithHgram{GruppeDBild}{GruppeDBild.gif - Original}{fig:gruppe_d_orig}

Das Ergebnis ist (Abb.~\ref{fig:gruppe_d_edit}) ist deutlich besser zu erkennen, hat alerdings eine sehr geringe Dynamik.
\imageWithHgram{GruppeDBild_edit}{GruppeDBild.gif - Angepasster Kontrast}{fig:gruppe_d_edit}



\subsection{Testbild: enhance-me.gif}
Das zweite Testbild ist in Abb.~\ref{fig:enhance_me_orig} zu sehen. Auch dieses Bild hat einen geringen Kontrast, allerdings ist gerade noch ein Gesicht erkennbar.
\imageWithHgram{enhance-me}{enhance-me.gif - Original}{fig:enhance_me_orig}

Das bearbeitete Bild (Abb.~\ref{fig:enhance_me_edit}) ist auch hier deutlich besser zu erkennen.
Allerdings gibt es viele weiße Punkte, welche wie Rauschen aussehen. Diese Punkte wurden durch die Kontrastanpassung ebenfalls verstärkt, wodurch die Verwedbarkeit des Bildes für weitere eventuell Schritte eigeschränkt ist.
\imageWithHgram{enhance-me_edit}{enhance-me.gif - Angepasster Kontrast}{fig:enhance_me_edit}


\subsection{Testbild: pluto.png}
Das letzte Testbild ist in Abb.~\ref{fig:pluto_orig} zu sehen. Im Gegensatz zu den vorherigen Bildern hat es einen hohen Kontrast und ist bereits gut zu erkennen.
\imageWithHgram{pluto}{pluto.png - Original}{fig:pluto_orig}

Diesmal ist das resultierende Bild (Abb.~\ref{fig:pluto_edit}) ähnlich gut zu erkennen das Original.
\imageWithHgram{pluto_edit}{pluto.png - Angepasster Kontrast}{fig:pluto_edit}

\section{Zusammenfassung}
Die Kontrastanpassung konnte in den meisten Fällen tatsächlich eine Verbesserung der Bilder vornehmen, hatte aber keine großen Auswirkungen auf Bilder, welche bereits einen hohen Kontrast hatten. Allerdings führten Orginale mit einem sehr geringen Kontrast zu Bildern mit einer sehr geringen Dynamik. Außerdem wurde auch das Rauschen in den Bildern verstärkt.




% conference papers do not normally have an appendix


% use section* for acknowledgement
%\section*{Acknowledgment}


%The authors would like to thank...





% trigger a \newpage just before the given reference
% number - used to balance the columns on the last page
% adjust value as needed - may need to be readjusted if
% the document is modified later
%\IEEEtriggeratref{8}
% The "triggered" command can be changed if desired:
%\IEEEtriggercmd{\enlargethispage{-5in}}

% references section

% can use a bibliography generated by BibTeX as a .bbl file
% BibTeX documentation can be easily obtained at:
% http://www.ctan.org/tex-archive/biblio/bibtex/contrib/doc/
% The IEEEtran BibTeX style support page is at:
% http://www.michaelshell.org/tex/ieeetran/bibtex/
%\bibliographystyle{IEEEtran}
% argument is your BibTeX string definitions and bibliography database(s)
%\bibliography{IEEEabrv,../bib/paper}
%
% <OR> manually copy in the resultant .bbl file
% set second argument of \begin to the number of references
% (used to reserve space for the reference number labels box)

TODO citations \cite{Fake:Buchkapitel, Sachs2006}%prevent errors

%
%\begin{thebibliography}{1}
%
%\bibitem{Fake:Buchkapitel}
%H.~Meier und L.~Müller, \emph{Zur Anwendung einer Referenz in der Literatur}, in ,,Wissenschaftlich arbeiten von Anfang an``, 1.~Aufl., Berlin, Springer, 2014.
%
%\bibitem{Fake:Wiss.Artikel}
%M.~Sugeno and T.~Tagaki, ''Fuzzy Identification of Systems and its Applications to Modeling and Control``, IEEE Trans. on Systems, Man \& Cybernetics, Vol.~15, pp.~116–132, 1985.
%
%
%\bibitem{IEEEhowto:kopka}
%H.~Kopka and P.~W. Daly, \emph{A Guide to \LaTeX}, 3rd~ed.\hskip 1em plus
%  0.5em minus 0.4em\relax Harlow, England: Addison-Wesley, 1999.
%
%
%\end{thebibliography}

\bibliography{Literature}
\bibliographystyle{deIEEEtran}



% that's all folks
\end{document}
